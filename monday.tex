\newpage
\section*{Monday Session Descriptions}\label{monday-descriptions}
\renewcommand{\conferenceDay}{\monday}
\setPageBackground
% time: Monday 09:30
% URL: https://pretalx.com/sotm2019/talk/LZ8HAH/

\newSmallTimeslot{09:30}
\abstractGHS{%
  Guillaume, Adrien Matissart \speakerAffiliation{Qwant},\linebreak
  J.~B. Griesner \speakerAffiliation{Qwant}%
}{%
  Qwant Maps: Geocode the World with OSM Data%
}{%
}{%
  A geocoder's purpose is to transform a textual user query into geographical coordinates.
  This task is complex for several reasons. Firstly, because the user query is freeform and may
  come with abbreviations, typography errors, different format standards etc. Secondly, no
  omniscient gazetteer exists to take into account of the complexity of the world
  administrative zones. We present some techniques used by Qwant Maps search engine to
  fulfil user needs.%
}

%%%%%%%%%%%%%%%%%%%%%%%%%%%%%%%%%%%%%%%%%%%

% time: Monday 09:30
% URL: https://pretalx.com/sotm2019/talk/V7QPGG/


\abstractHSO{%
  Tobias Knerr%
}{%
  OSM2World: 3D OSM in Your Browser%
}{%
}{%
  OSM2World is an open-source renderer capable of creating rich 3D worlds from OpenStreetMap data.
  It is now available as a web application pow\-ered by WebGL.%
}

%%%%%%%%%%%%%%%%%%%%%%%%%%%%%%%%%%%%%%%%%%%

% time: Monday 09:30
% URL: https://pretalx.com/sotm2019/talk/LHSWEF/

\enlargethispage{1\baselineskip}
\abstractKHS{%
  Jessica Bergmann%
}{%
  The Next Generation of Mappers:\linebreak Learning from YouthMappers~\noVideo%
}{%
}{%
  What does it take to build a new generation of mappers ready to solve
  challenges within their community? This presentation will feature students
  from YouthMappers sharing how they are deciding what issues to map,
  engaging and training new mappers, especially women, and using map data to
  advocate for change in their local and global communities. Learn how you
  can plan your next mapping initiative with youth at the forefront.

  The following YouthMappers will speak during this session:
  %TODO adapt space
  %\vspace{-2em}
  \begin{itemize}
    \RaggedRight
    \setlength{\itemsep}{0pt plus 0pt}
    \item Diana Carolina Ortega Espinosa
    \item Confidence Kpodo
    \item Aman K C
    \item Micheal Kaluba
    \item Erneste Ntakobangize
  \end{itemize}%
  \justifying
}
\vspace{-0.5\baselineskip}

%%%%%%%%%%%%%%%%%%%%%%%%%%%%%%%%%%%%%%%%%%%

% time: Monday 09:30
% URL: https://pretalx.com/sotm2019/talk/GUUUYW/

\abstractMathematikonC{%
  Eugene Alvin Villar%
}{%
  OpenStreetMap and Wikidata:\linebreak Awesome Together~\noVideo%
}{%
}{%
  A workshop aiming to introduce Wikidata and how to contribute to it, to explain how Wikidata and
  OpenStreetMap are currently connected, and to discuss issues and ways for the OSM and Wikidata communities
  to further collaborate together.%
}


%%%%%%%%%%%%%%%%%%%%%%%%%%%%%%%%%%%%%%%%%%%

% time: Monday 10:00
% URL: https://pretalx.com/sotm2019/talk/ARMCT7/

\newSmallTimeslot{10:00}
\abstractGHS{%
  Sebastian Ritterbusch \speakerAffiliation{iXpoint Informationssysteme}%
}{%
  Routing for Humans%
}{%
}{%
  The OpenStreetMap provides probably the most comprehensive digital path network for pedestrians,
  which had been largely unused in its actual depth of detail so far. Based on findings from the
  TERRAIN project, this talk will go into detail how this network can be used for roadside-aware
  pedestrian navigation efficiently, and what challenges and mappings problems occurred.
  Furthermore, specific needs of various person groups will also have to be considered facing
  variable data quality.%
}

\enlargethispage{2\baselineskip}
%%%%%%%%%%%%%%%%%%%%%%%%%%%%%%%%%%%%%%%%%%%

% time: Monday 10:00
% URL: https://pretalx.com/sotm2019/talk/LBREW8/

\abstractGHS{%
  Shinji Enoki%
}{%
  OpenDatathon Activities in Japan%
}{%
}{%
  As one of the outreach activities in Japan, OpenDatathon regularly holds mapping parties
  simultaneously with Wikipedia editing events.  As a member of an organiser group, I will
  introduce OpenDatathon case studies and consider their potential.%
}

%%%%%%%%%%%%%%%%%%%%%%%%%%%%%%%%%%%%%%%%%%%

% time: Monday 10:30
% URL: https://pretalx.com/sotm2019/talk/9FQVUT/

\newSmallTimeslot{10:30}
\abstractGHS{%
  Sam Milsom \speakerAffiliation{Open Data Manchester}%
}{%
  Mapping Mobility in Stockport%
}{%
  Creating a mobility map of Stockport town centre
}{%
  In early 2019, Open Data Manchester teamed up with Stockport Council, Disability Stockport
  and Age UK Stockport to crowdsource data around mobility and accessibility in the town centre.
  This talk will cover the mapping methodology, findings, difficulties and solutions, as well as
  suggesting ways in which we can better map for these communities, making the data collected for
  OSM more inclusive and accessible for all.%
}

%%%%%%%%%%%%%%%%%%%%%%%%%%%%%%%%%%%%%%%%%%%

% time: Monday 10:30
% URL: https://pretalx.com/sotm2019/talk/XQUHP3/


\abstractHSO{%
}{%
  Lightning Talks VI%
}{%
}{%
  Come to the welcome desk in the Chemie-Hörsaalgebäude to sign up. See the board there for an up-to-date schedule.
}

%%%%%%%%%%%%%%%%%%%%%%%%%%%%%%%%%%%%%%%%%%%

% time: Monday 10:30
% URL: https://pretalx.com/sotm2019/talk/ADYMVF/


\abstractKHS{%
  Hrvoje Bogner \speakerAffiliation{OpenStreetMap Croatia}%
}{%
  OpenStreetMap in Croatia~\noVideo%
}{%
}{%
  % commented out repetition of title
  %State of OpenStreetMap in Croatia.
  Community that builds the data, and community that uses the
  data.%
}

%%%%%%%%%%%%%%%%%%%%%%%%%%%%%%%%%%%%%%%%%%%

% time: Monday 11:30
% URL: https://pretalx.com/sotm2019/talk/F9D8QG/

\enlargethispage{1\baselineskip}
\newSmallTimeslot{11:30}
\abstractGHS{%
  Martin Dittus \speakerAffiliation{Oxford Internet Institute}, David Garcia \speakerAffiliation{University of Canterbury}%
}{%
  Caretography~-- Mapping Difficult\linebreak Issues with OSM during Difficult Times%
}{%
}{%
  We map because we care to represent the world. Yet maps are never ``true'', they are shaped by their
  creators and their circumstances. Map-making is world-making: Maps by different authors can give
  access to different worlds. So how can we make, share, and use maps that are created \emph{by} these
  worlds, and not just by a privileged few? How can vulnerable communities influence how they’re
  represented and affected by our maps?%
}

%%%%%%%%%%%%%%%%%%%%%%%%%%%%%%%%%%%%%%%%%%%

% time: Monday 11:30
% URL: https://pretalx.com/sotm2019/talk/JGHWKY/


\abstractHSO{%
  Adrien Pavie%
}{%
  Integrating and Validating Open Data in OSM Using Street Pictures%
}{%
  Integrating Open Data Remotely, the Right and Easy Way
}{%
  Pic4Review now helps contributors to integrate open data properly in OpenStreetMap using street
  pictures validation. Discover the way it works and how this can help improve both OSM and open
  datasets.%
}

%%%%%%%%%%%%%%%%%%%%%%%%%%%%%%%%%%%%%%%%%%%

% time: Monday 11:30
% URL: https://pretalx.com/sotm2019/talk/PPTHFQ/

\enlargethispage{1\baselineskip}
\abstractKHS{%
  Roland Olbricht%
}{%
  New Processes to Agree on Tagging Suggestions and Their Interaction with the Editing Software Available\linebreak on openstreetmap.org~\noVideo%
}{%
}{%
  Tags are only helpful if one can understand what they mean.
  However, it is rather difficult to figure out the tag semantics,
  and there is no agreement how to agree on the semantics.
  The public communication channels mailing list and wiki discussions both
  have relatively low attendance.
  An incident gave rise to the rumour that in fact editing software
  developers were unilaterally controlling tags.

  In this session the results of an open discussion before the
  conference will be presented,
  and attendees can contribute further thoughts during the event.
  The basic idea is to prioritise what most likely helps to get feedback
  from much more of the mappers using the tags.
  If you want to contribute beforehand, please send your suggestions to
  the talk@ or tagging@ mailing list, or email the host directly
  at roland.olbricht@gmx.de.
}

%%%%%%%%%%%%%%%%%%%%%%%%%%%%%%%%%%%%%%%%%%%

% time: Monday 11:30
% URL: https://pretalx.com/sotm2019/talk/QHTNEY/


\abstractMathematikonC{%
  Harry Mahardhika Machmud%
}{%
  Custom Presets Creation Using JOSM~\noVideo%
}{%
}{%
  \emph{Experienced mappers only}

  \noindent Creating presets sometimes becomes a hassle if you do not have
  experience reading the XML language. There is a simple method to create presets using a
  JOSM plugin. This workshop will teach how to create presets using JOSM.%
}

%%%%%%%%%%%%%%%%%%%%%%%%%%%%%%%%%%%%%%%%%%%

% time: Monday 12:00
% URL: https://pretalx.com/sotm2019/talk/LBGPCD/

\newSmallTimeslot{12:00}
\abstractHSO{%
  Victor Shcherb \speakerAffiliation{OsmAnd}%
}{%
  Introduce OpenPlaceReviews and Connect to OpenStreetMap%
}{%
  OpenPlaceReviews~-- Open, Collaborative, Trustworthy
}{%
  As of today we have OpenStreetMap but it doesn't fit all data and some data is not recommended for
  submission. We've got user reviews request in OsmAnd and Maps.Me and we would like to collaborate
  with OpenStreetMap community to create independent open platform for reviews.%
}

%%%%%%%%%%%%%%%%%%%%%%%%%%%%%%%%%%%%%%%%%%%

% time: Monday 12:30
% URL: https://pretalx.com/sotm2019/talk/7FBMMM/

\newSmallTimeslot{12:30}
\abstractGHS{%
  Davey Lovin%
}{%
  Access to Prosperity: Quantifying Infrastructure Impact with OSM%
}{%
}{%
  In many regions of the world a population’s access to essential services is unduly constrained by
  a lack of proper infrastructure. By performing accessibility analysis using OSM data, we can
  understand how route infrastructure impacts access to essential services and use that information
  to inform an intervention.

  This talk explores accessibility analysis performed to understand the impact of footbridge
  construction in Eswatini and introduces a python framework enabling users to perform similar
  analyses.%
}

%%%%%%%%%%%%%%%%%%%%%%%%%%%%%%%%%%%%%%%%%%%

% time: Monday 12:30
% URL: https://pretalx.com/sotm2019/talk/MYFAAH/


\abstractHSO{%
}{%
  Lightning Talks VII%
}{%
}{%
  Come to the welcome desk in the Chemie-Hörsaalgebäude to sign up. See the board there for an up-to-date schedule.
}

%%%%%%%%%%%%%%%%%%%%%%%%%%%%%%%%%%%%%%%%%%%

% time: Monday 14:00
% URL: https://pretalx.com/sotm2019/talk/AT9YPK/

\newSmallTimeslot{14:00}
\abstractGHS{%
  Antoine Riche \speakerAffiliation{Carto'Cité, SNCF}%
}{%
  Pedestrian Routing in Complex Areas%:
  %the Case of Paris Railway Stations%
}{%
  The Case of Paris Railway Stations%
}{%
  Have you ever been lost inside a gigantic railway station? SNCF, the French railway company, is
  developing a pedestrian routing and navigation service to help travelers find their way inside and
  around railway stations. This talk exposes the challenges and how they have been addressed to
  provide a robust solution that can handle the great variety of data as well as routing through
  open spaces.%
}

%%%%%%%%%%%%%%%%%%%%%%%%%%%%%%%%%%%%%%%%%%%

% time: Monday 14:00
% URL: https://pretalx.com/sotm2019/talk/CXVMJ8/


\abstractHSO{%
  Johan Wiklund \speakerAffiliation{Entur AS}%
}{%
  Norway: Successful Deployment of OSM in Public Transport%
}{%
}{%
  Norway has deployed OSM on a national level for journey planning and is taking initiatives to
  expand the user base and usage of OSM in Norway and abroad.%
}

%%%%%%%%%%%%%%%%%%%%%%%%%%%%%%%%%%%%%%%%%%%

% time: Monday 14:00
% URL: https://pretalx.com/sotm2019/talk/ZTCEZE/


\abstractKHS{%
  Zacharia Muindi \speakerAffiliation{Map Kibera Trust}%
}{%
  Participatory Mapping for Open\linebreak Counties~\noVideo%
}{%
}{%
  In Kenya most County Participatory budgeting (an intensive annual participatory budgeting process
  supported by World Bank) sessions rely on memory or on hand-drawn paper maps of existing terrain
  and features in order to determine where they should place new water points, health centres, and
  other key new projects.%
}

%%%%%%%%%%%%%%%%%%%%%%%%%%%%%%%%%%%%%%%%%%%

% time: Monday 14:00
% URL: https://pretalx.com/sotm2019/talk/RUJWP8/


\abstractMathematikonC{%
  Beata Tautan-Jancso, Laura Dumitru \speakerAffiliation{Telenav}%
}{%
  ImproveOSM%~-- MissingRoads Workshop~\noVideo%
}{%
  MissingRoads Workshop~\noVideo%
}{%
  In this workshop, OSM mappers and software developers Beata and Laura will help you get started
  adding the more than two million roads that are missing from OSM according to ImproveOSM, the global
  open dataset that compares actual car trips with OSM data.%
}


%%%%%%%%%%%%%%%%%%%%%%%%%%%%%%%%%%%%%%%%%%%

% time: Monday 14:30
% URL: https://pretalx.com/sotm2019/talk/ZJYT7F/

\newSmallTimeslot{14:30}
\abstractGHS{%
  Frédéric Rodrigo \speakerAffiliation{Makina Corpus}%
}{%
  Automatically Annotate a Pedestrian Route with OSM Landmarks%
}{%
}{%
  Replacing the classic ``Continue for about 200 m then turn left'' by guidance instructions in more
  natural language automated from landmarks are still a research topic but aims to allow users to
  move with more confidence. The objective is to test it on an indoor/outdoor pedestrian route
  calculator. Anything that can be used as a landmark is extracted from OpenStreetMap, then
  categorised and classified to annotate the route: relevance, visibility, relative position etc.%
}

%%%%%%%%%%%%%%%%%%%%%%%%%%%%%%%%%%%%%%%%%%%

% time: Monday 14:30
% URL: https://pretalx.com/sotm2019/talk/NWW7GF/


\abstractHSO{%
  Victor Shcherb \speakerAffiliation{OsmAnd}, Eugene Kizevich%
}{%
  Public Transport Navigation Using\linebreak OpenStreetMap by OsmAnd%
}{%
}{%
  This presentation describes the stages of developing and starting a new feature in the OsmAnd app.
}



%%%%%%%%%%%%%%%%%%%%%%%%%%%%%%%%%%%%%%%%%%%

% time: Monday 15:00
% URL: https://pretalx.com/sotm2019/talk/EKQBYN/

\newSmallTimeslot{15:00}
\abstractGHS{%
  Denis Cheynet \speakerAffiliation{SNCF}%
}{%
  From Car Routing to Train Routing~\noVideo%
}{%
}{%
  Adaptation of GraphHopper (routing and map matching) to the railway context in order to serve a
  vast diversity of usages.%
}

%%%%%%%%%%%%%%%%%%%%%%%%%%%%%%%%%%%%%%%%%%%

% time: Monday 15:00
% URL: https://pretalx.com/sotm2019/talk/RYPLFZ/


\abstractHSO{%
  Jiri Komarek \speakerAffiliation{MapTiler}%
}{%
  OSM Vector Tiles in Custom Coordinate Systems%
}{%
}{%
  OpenMapTiles is an open-source set of tools for processing OSM data into vector maps,
  which can be produced in various coordinate systems.%
}

%%%%%%%%%%%%%%%%%%%%%%%%%%%%%%%%%%%%%%%%%%%

% time: Monday 15:00
% URL: https://pretalx.com/sotm2019/talk/UELRPR/


\abstractKHS{%
  Tasauf A Baki Billah \speakerAffiliation{OpenStreetMap Bangladesh Foundation, Bangladesh Open Innovation Lab}%
}{%
  State of the Map Bangladesh%: Transforming a Resilient Community by Institutionalising OSM
  \noVideo%
}{%
  Transforming a Resilient Community by Institutionalising OSM%
}{%
  The story of community-based resilience activities in Bangladesh evol\-ving into the institutionalisation
  of OSM and moving forward with a vision to ensure a sustainable open geodata ecosystem in
  Bangladesh.%
}

%%%%%%%%%%%%%%%%%%%%%%%%%%%%%%%%%%%%%%%%%%%

% time: Monday 16:00
% URL: https://pretalx.com/sotm2019/talk/SWAGX7/

\newSmallTimeslot{16:00}
\abstractGHS{%
}{%
  Closing%
}{%
}{%
}

%%%%%%%%%%%%%%%%%%%%%%%%%%%%%%%%%%%%%%%%%%%
