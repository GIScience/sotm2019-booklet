\newpage
\section*{Sunday Session Descriptions}\label{sunday-descriptions}
\renewcommand{\conferenceDay}{\sunday}
\setPageBackground
\newSmallTimeslot{09:00}
\abstractAcademic{%
  A. Yair Grinberger \speakerAffiliation{Heidelberg University},
  Marco Minghini \speakerAffiliation{European Commission, Joint Research Centre},
  Levente Juhász \speakerAffiliation{Florida International University},
  Peter Mooney \speakerAffiliation{Maynooth University},
  Godwin Yeboah\linebreak \speakerAffiliation{University of Warwick}
}{%
  Bridging the Map?
  Exploring Interactions between the Academic and Mapping Communities in OpenStreetMap
}{%
}{%
  This talk presents an initial inquiry into the relations between the academic and mapping
  communities in OpenStreetMap, based on a review of recent publications, interviews of colleagues,
  and self-reflection of the authors. By this, we aim to understand how and when research-community
  interactions come to be, what is their nature, and how these can be improved and made more
  productive for both sides.
}


% time: Sunday 09:30
% URL: https://pretalx.com/sotm2019/talk/P8AG7K/

\newSmallTimeslot{09:30}
\abstractGHS{%
  Peter Karich \speakerAffiliation{GraphHopper GmbH}%
}{%
  Flexible Routing with GraphHopper%
}{%
}{%
  In this talk we give an overview on how to use GraphHopper to provide more flexible routing
  (based on weather information, road class, road width etc.) and how this could be also used for
  visualisation purposes or data analysis.%
}

%%%%%%%%%%%%%%%%%%%%%%%%%%%%%%%%%%%%%%%%%%%

% time: Sunday 09:30
% URL: https://pretalx.com/sotm2019/talk/GQ3AAF/

\enlargethispage{1\baselineskip}
\abstractHSO{%
  Nicolas Chavent, Séverin Menard \speakerAffiliation{LLG/ProjetEOF}%
}{%
  ``Mapathon, Mapathon, Mapathon!''%
}{%
}{%
  Who benefits from mapathons? Between (over)communication and (over)\-attribution, critical
  feedback on the inflation of a form of action oversold in the field of humanitarian action and
  development aid. Do they really benefit OSM?%
}

\abstractAcademic{%
  Dipto Sarkar \speakerAffiliation{National University of Singapore},\linebreak So Hoi Kay \speakerAffiliation{ibid.}
}{%
  OpenStreetMap as a Space~\noVideo
}{%
}{%
  Can OSM be used in classroom teaching as a definitive instance of a geo\-graphic space? We provide
  instances of how OSM as space may be of interest to researchers and for students of geography
  alike. Concepts of human geography manifested in OSM can be used to understand how digital
  geographies and offline activities are intrinsically interwoven with and influenced by each other.
  Thus, introduction to digital geography in classrooms can be through concepts in human geography.
}

%%%%%%%%%%%%%%%%%%%%%%%%%%%%%%%%%%%%%%%%%%%

% time: Sunday 09:30
% URL: https://pretalx.com/sotm2019/talk/PRHPXV/

\enlargethispage{1\baselineskip}
\abstractKHS{%
}{%
  Scholar Lightning Talks~\noVideo%
}{%
}{%
  The following lightning talks will be presented in this session:
  \begin{itemize}
    \RaggedRight
    \setlength{\itemsep}{0pt plus 0pt} % Aufzählungspunktabstand auf 0
    \lightningTalk{Mapping Advocacy: Breastfeeding stations in the Philippines}{Jen Alconis Ayco}
    \lightningTalk{Open data for disaster governance in Nepal}{Pradip Khatiwada \speakerAffiliation{Youth Innovation Lab}}
    \lightningTalk{MapLesotho}{Tshedy Letsie}
    \lightningTalk{State of the Map~-- India}{Naveen Francis}
    \lightningTalk{OSM for Gambia (mapping the whole Gambia)}{Jariatou Jallow \speakerAffiliation{Connected YouthMappers}}
    \lightningTalk{HOT and OSM in Tanzania, the ups and downs of crowdsourced mapping}{Emmor Nile}
    \lightningTalk{Community cartographers}{Fernando Castro T \speakerAffiliation{Fundación OSM Colombia}}
    \lightningTalk{Using OpenStreetMap building footprints data for population distribution model: a case study in Cavite, Philippines}{Dinnah Feye H. Andal}
    \lightningTalk{Public transport in OSM}{Porfirio Carrasco}
    \lightningTalk{Rendering order in openstreetmap-carto}{Matthijs Melissen}
  \end{itemize}%
  \justifying
}

%%%%%%%%%%%%%%%%%%%%%%%%%%%%%%%%%%%%%%%%%%%

% time: Sunday 10:00
% URL: https://pretalx.com/sotm2019/talk/TVSA9F/

\newSmallTimeslot{10:00}
\abstractGHS{%
  Kevin Bullock \speakerAffiliation{Maxar}%
}{%
  Imagery Solutions in OpenStreetMap%
}{%
}{%
  Satellite imagery has materially enhanced OpenStreetMap and improved editing and validation. In
  this talk we present recent enhancements we have made to get even more information from
  satellite imagery.%
}

%%%%%%%%%%%%%%%%%%%%%%%%%%%%%%%%%%%%%%%%%%%

% time: Sunday 10:00
% URL: https://pretalx.com/sotm2019/talk/VWBV88/


\abstractHSO{%
  Séverin Menard, Nicolas Chavent \speakerAffiliation{LLG/ProjetEOF}%
}{%
  OSMF Local Chapters in Countries of the Global South
}{%
  What Can We Learn from OSM associations Dynamics in French Speaking Southern Countries of Africa and the Caribbean?%
}{%
  This talk will share lessons that OSM and OSMF members can learn about the multi-year collective
  dynamics around OpenStreetMap which unfold in French-speaking southern countries of Africa and the
  Caribbean with the view to identifying paths for local OSM grassroots groups' evolution toward becoming
  formal OSMF Local Chapters.%
}

\abstractAcademic{%
  Toshikazu Seto \speakerAffiliation{University of Tokyo}, Hiroshi Kanasugi,
  \speakerAffiliation{ibid.}, Yuichiro Nishimura \speakerAffiliation{Nara Women's University}
}{%
  Analysis of OSM Data through OSM Notes User Posting
}{%
}{%
  In this research, the OSM Notes feature is mainly viewed as data that can be examined speedily from global OSM data in terms of the content of the notes posted and the location of users.
}

\newpage

%%%%%%%%%%%%%%%%%%%%%%%%%%%%%%%%%%%%%%%%%%%

% time: Sunday 10:30
% URL: https://pretalx.com/sotm2019/talk/EGMAVR/

\newSmallTimeslot{10:30}
\abstractGHS{%
}{%
  Lightning Talks III%
}{%
}{%
  The following lightning talks will be presented in this session:
  %TODO adapt space
  \begin{itemize}
    \RaggedRight
    \setlength{\itemsep}{0pt plus 0pt}
    \lightningTalk{Human in the Loop: Verifying machine-generated data for better maps}{Katrin Humal \speakerAffiliation{Mapillary}}
    \lightningTalk{Share the word}{Ilya Zverev \speakerAffiliation{Juno Lab}}
    \lightningTalk{Enhancing OSM with missing roads}{Beata Tautan-Jancso \speakerAffiliation{Telenav}}
    \lightningTalk{Community-led mapping helping in policy changes}{Sibabrata Choudhury \speakerAffiliation{Envision}}
    \lightningTalk{How to create a data annotation process used for navigation}{Alina Negreanu}
  \end{itemize}%
  \justifying
}

%%%%%%%%%%%%%%%%%%%%%%%%%%%%%%%%%%%%%%%%%%%

% time: Sunday 10:30
% URL: https://pretalx.com/sotm2019/talk/7Q97AH/


\abstractHSO{%
  Ramya Ragupathy \speakerAffiliation{HOT}, Wille Marcel Lima Malheiro, Felix Delattre \speakerAffiliation{HOT}%
}{%
  Tales from the Tasking Manager%
}{%
}{%
  The Tasking Manager is OpenStreetMap’s most used software tool to organise mapathons, community
  mapping initiatives, and professional mapping teams. Over the last year it has been developed
  further significantly. This talk will give an update on the newest developments and the emerging
  community around the application.%
}

\abstractAcademic{%
  Peter Mooney \speakerAffiliation{Maynooth University}
}{%
  A Novel Application of Models of Species Abundance to Better Understand OSM Community Structure and Interactions
}{%
}{%
  The OSM community is a global community crossing cultures, languages, and geographical boundaries. Researchers have been working to develop automated approaches to understanding the composition of this community through their contributions to the OSM database. In this talk we propose a new and novel application of theories and models of species abundance from ecological science to understand contributor community structure and distributions in OSM.
}


%%%%%%%%%%%%%%%%%%%%%%%%%%%%%%%%%%%%%%%%%%%

% time: Sunday 11:30
% URL: https://pretalx.com/sotm2019/talk/YFUEBH/

\newSmallTimeslot{11:30}
\abstractGHS{%
  Paul Norman%
}{%
  Overview of Map Serving Architectures%
}{%
}{%
  Serving maps is one of the most common uses of OpenStreetMap data. This talk goes over the
  component-level architecture for common and uncommon ways to serve maps.%
}

%%%%%%%%%%%%%%%%%%%%%%%%%%%%%%%%%%%%%%%%%%%

% time: Sunday 11:30
% URL: https://pretalx.com/sotm2019/talk/YENWFX/


\abstractHSO{%
  Alison Moon \speakerAffiliation{thinkWhere}%
}{%
  ``Our Falkirk''
}{%
  Mitigating the Impacts of Poverty using OSM Data Themes%
}{%
  Services that provide money advice, access to food provision, digital access, and community support
  are key to supporting people facing poverty.  ``Our Falkirk'' is a simple mapping platform for
  service dis\-covery that allows enriched OSM data to be easily described, mapped and shared through
  the concept of data ‘themes’.%
}

\enlargethispage{1\baselineskip}
\abstractAcademic{%
  Rui Zhang \speakerAffiliation{IBM Research},
  Marcus Freitag \speakerAffiliation{ibid.},
  Conrad Albrecht \speakerAffiliation{ibid.},
  Siyuan Lu \speakerAffiliation{ibid.},
  Wei Zhang \speakerAffiliation{ibid.}
}{%
  Towards Scalable Geospatial Remote Sensing for Efficient OSM Labeling~\noVideo
}{%
}{%
  The time OpenStreetMap mappers invest in labeling the world is valuable. We present how methods from remote
  sensing, big data distributed computing, and artificial intelligence can be combined to support
  human analysis of geo-spatial data.
}


%%%%%%%%%%%%%%%%%%%%%%%%%%%%%%%%%%%%%%%%%%%

% time: Sunday 11:30
% URL: https://pretalx.com/sotm2019/talk/NCSSPK/


\abstractKHS{%
  Nicolas Chavent \speakerAffiliation{LLG/ProjetEOF}, Séverin Ménard
}{%
  Bilingual Breakout Session~\noVideo%
}{%
}{%
  This session will present the rise of active, self-standing grassroots communities in Haiti,
  Western and Central Africa since 2010 resulting from a unique set of continued support actions
  replicable in other territories, by an ensemble of speakers.
  %% commented out because it makes no sense to repeat their origin. The list was added later.
  %from (at least) France, Burkina Faso,
  %Senegal and Togo.

  The following people will speak during this session:
  %TODO adapt space
  %\vspace{-2em}
  \begin{itemize}
    \RaggedRight
    \setlength{\itemsep}{0pt plus 0pt}
    \item Nicolas Chavent (France)
    \item Séverin Ménard (France)
    \item Amadou Ndong (Senegal)
    \item Aimée Sama (Togo)
    \item Innocent Dibloni (Burkina Faso)
    \item Saliou Abdou (Benin)
    \item Racky Ly (Ivory Coast)
  \end{itemize}%
  \justifying
}

%%%%%%%%%%%%%%%%%%%%%%%%%%%%%%%%%%%%%%%%%%%

% time: Sunday 11:30
% URL: https://pretalx.com/sotm2019/talk/3Z3RD9/


\abstractMathematikonC{%
  Angjelina Dervishaj%
}{%
  First Steps with OSM Editors~\noVideo%
}{%
}{%
  During this session I will share some tips and tools for newcomers to start making their first edits
  on OpenStreetMap. I will do so by explaining the mapping concepts, process, the use of editors, and
  will continue with the practicing part. By explaining this process, I aim to make it easier for
  new contributors to get started with editing and will show some everyday examples to illustrate
  the importance of even the smallest contributions.%
}

%%%%%%%%%%%%%%%%%%%%%%%%%%%%%%%%%%%%%%%%%%%

% time: Sunday 12:00
% URL: https://pretalx.com/sotm2019/talk/PJE8GK/

\enlargethispage{1\baselineskip}
\newSmallTimeslot{12:00}
\abstractGHS{%
  Sarah Hoffmann%
}{%
  Customising Search for Special-Interest Maps%
}{%
}{%
  This talk discusses different ways how to improve the search experience for domain-specific maps.%
}

%%%%%%%%%%%%%%%%%%%%%%%%%%%%%%%%%%%%%%%%%%%

% time: Sunday 12:00
% URL: https://pretalx.com/sotm2019/talk/3BZCXA/


\abstractHSO{%
  Pierre Béland%
}{%
  OSM Quality Mapping: Metrics to Monitor Buildings Outbounds%
}{%
  How Can We Better Monitor and Correct Quality Problems?
}{%
  Mapathons and imports present a quality challenge for the OSM community. This presentation
  focuses on buildings. It presents metrics and shows progress of tool development to monitor and
  correct quality problems in the OSM database or before importing.%
}


\abstractAcademic{%
  Nikola Milojevic-Dupont \speakerAffiliation{Technische Universität Berlin},
  Peter-Paul Pichler \speakerAffiliation{Potsdam Institute for Climate Impact Research},
  Lynn H. Kaack \speakerAffiliation{ETH Zurich},
  Steffen Lohrey \speakerAffiliation{Technische Universität Berlin},
  Felix Creutzig \speakerAffiliation{ibid.}
}{%
  Estimating Latent Energy Demand of Buildings
}{%
}{%
  We propose a model that uses only open data for estimating the minimal energy use of individual
  buildings for heating and cooling at scale. The workflow is divided in two main blocks: (i)
  predicting at scale a 3D building stock using OpenStreetMap data, and (ii) estimating the energy
  use of buildings individually with a back-end model.
}


%%%%%%%%%%%%%%%%%%%%%%%%%%%%%%%%%%%%%%%%%%%

% time: Sunday 12:30
% URL: https://pretalx.com/sotm2019/talk/7LMH8R/

\newSmallTimeslot{12:30}
\abstractGHS{%
  Thomas Skowron%
}{%
  Is Your OSM App Spying on You?%
}{%
}{%
  OpenStreetMap enables people to use third-party apps that seem to be more suitable for
  privacy-conscientious users, but are we as users really private when using OSM-based apps?%
}

%%%%%%%%%%%%%%%%%%%%%%%%%%%%%%%%%%%%%%%%%%%

% time: Sunday 12:30
% URL: https://pretalx.com/sotm2019/talk/9SSZQH/


\abstractHSO{%
  David Manzer, Matthew Gibb \speakerAffiliation{Radiant Solutions},\linebreak Clarisse Abalos \speakerAffiliation{Radiant Solutions}%
}{%
  Bringing Validation to Users: Integrating Quality Assurance Checks into Map Editors%
}{%
}{%
  Providing more validation checks with MapRules and MapCSS tag checks in iD and JOSM to direct
  mappers to issues as they map. As well as using Overpass queries to retrieve features with data
  quality issues.%
}


\abstractAcademic{%
  Harm Delva \speakerAffiliation{Ghent University},
  Julián Rojas \speakerAffiliation{ibid.},
  Ben Abelshausen \speakerAffiliation{Open Knowledge Belgium},
  Pieter Colpaert \speakerAffiliation{Ghent University},
  Ruben Verborgh \speakerAffiliation{ibid.}
}{%
  Client-Side Route Planning:
  Preprocessing the OpenStreetMap Road Network for Routable Tiles
}{%
}{%
  Travellers have high expectations of their route planners. We explore how preprocessing techniques
  applied to linked open data derived from OSM (Routable Tiles) can provide satisfying performance
  for client-side route planning.
}

%%%%%%%%%%%%%%%%%%%%%%%%%%%%%%%%%%%%%%%%%%%

% time: Sunday 14:00
% URL: https://pretalx.com/sotm2019/talk/88ZHKQ/

\newSmallTimeslot{14:00}
\abstractGHS{%
  Vincent Privat%
}{%
  What's behind JOSM?%
}{%
  Presentation of the JOSM Development Model: How Is It Made? By Whom? How Can I Help?
}{%
  JOSM is almost as old as OSM but few people really know what it takes to maintain your preferred
  editor.

  We will present the development model of JOSM and who is part of its active community: developers,
  translators; testers; plugin authors; end users; sponsors; etc.

  We will talk about project difficulties, the major achievements made in the past years, what
  work is currently in progress, and what will happen in the near future!%
}

%%%%%%%%%%%%%%%%%%%%%%%%%%%%%%%%%%%%%%%%%%%

% time: Sunday 14:00
% URL: https://pretalx.com/sotm2019/talk/CAD93S/


\abstractHSO{%
  Felix Kunde%
}{%
  Spatial Indexes for OSM in PostGIS%
}{%
}{%
  Indexing OpenStreetMap data in your PostGIS database for fast spatial queries is not as straightforward as
  one might hope. And with each release of PostgreSQL/PostGIS there are more options to try.
  This talk will explain different spatial indexing concepts and best practices in PostGIS and
  present some benchmarking results.%
}


\abstractAcademic{%
  Marco Minghini \speakerAffiliation{European Commission, Joint Research Centre},
  Daniele Oxoli \speakerAffiliation{Politecnico di Milano},
  Francesco Frassinelli \speakerAffiliation{Norsk institutt for naturforskning},\linebreak
  Maria Antonia Brovelli \speakerAffiliation{Politecnico di Milano}
}{%
  Intrinsic Assessment of OpenStreetMap Contribution Patterns through\linebreak Exploratory Spatial Data Analysis
}{%
}{%
  This study adopts a statistical approach based on Exploratory Spatial Data Analysis to identify
  underlying contribution patterns of OpenStreetMap. Univariate and multivariate analyses on a
  number of variables computed from OSM history on a regular hexagonal grid in Milan, Italy allow us
  to detect a number of local clusters and local outliers, which shed light on the complexity
  of OSM temporal evolution driven by active local contributors and communities, data imports, and
  mapping parties.
}

%%%%%%%%%%%%%%%%%%%%%%%%%%%%%%%%%%%%%%%%%%%

% time: Sunday 14:00
% URL: https://pretalx.com/sotm2019/talk/Z7L9HF/


\abstractKHS{%
  Patricia Solis \speakerAffiliation{YouthMappers/Arizona State University}, Miriam Gonzalez \speakerAffiliation{GeochicasOSM}, Heather Leson%
}{%
  Diversity and Inclusion in OSM~\noVideo%
}{%
}{%
  The OSM community is global and diverse. Building on last year's Open Heroines conversation we
  will co-create a space for OSM to talk about how to improve diversity and inclusion in our amazing
  project. All welcome.%
}

%%%%%%%%%%%%%%%%%%%%%%%%%%%%%%%%%%%%%%%%%%%

% time: Sunday 14:00
% URL: https://pretalx.com/sotm2019/talk/WSAFMM/


\abstractMathematikonC{%
  Wille Marcel Lima Malheiro, Andrey Golovin \speakerAffiliation{Mapbox}%
}{%
  Using OSMCha to Understand Bad\linebreak Edits~\noVideo%
}{%
}{%
  Protecting OSM is a continuous process performed by Mapbox to secure maps from
  displaying erroneous edits. Any edits that raise suspicion are flagged in OSMCha, an open service
  that allows us to check low-quality changes that are made by the members of the OSM project in a
  shared database. This not only helps to report our findings to the community but examine them in
  aggregate and draw conclusions to improve our data quality processes.%
}

%%%%%%%%%%%%%%%%%%%%%%%%%%%%%%%%%%%%%%%%%%%

% time: Sunday 14:30
% URL: https://pretalx.com/sotm2019/talk/EBHGTW/

\newSmallTimeslot{14:30}
\abstractGHS{%
  Jerry Clough, Dan Stowell%
}{%
  Mapping Solar Panels Can Save\linebreak Megatons of CO\textsubscript{2}%
}{%
}{%
  We are working to map all the solar panels (photovoltaic, ``PV'') in the world. Why? The data can be
  used directly to reduce carbon emissions from power generation. We will share our experiences of
  surveying, aerial mapping, and machine vision to find all the hundreds of thousands of solar panels
  in our countries.%
}

%%%%%%%%%%%%%%%%%%%%%%%%%%%%%%%%%%%%%%%%%%%

% time: Sunday 14:30
% URL: https://pretalx.com/sotm2019/talk/RLZJG9/


\abstractHSO{%
}{%
  Lightning Talks IV%
}{%
}{%
  Come to the welcome desk in the Chemie-Hörsaalgebäude to sign up. See the board there for an up-to-date schedule.
}


\abstractAcademic{%
  Jennings Anderson \speakerAffiliation{University of Colorado Boulder},
  Dipto Sarkar \speakerAffiliation{National University of Singapore},
  Leysia Palen \speakerAffiliation{University of Colorado Boulder}
}{%
  Corporate Editors in the Evolving Landscape of OSM:
  A Close Investigation of the Impact to the Map and Community
}{%
}{%
  More than 17 million edits globally have been made by paid contributors in the last five years. We
  look at the long history of corporate involvement in OpenStreetMap and dig into the data to quantify
  the impact this latest evolution of corporate involvement is having on the map and explore the
  interactions between paid and volunteer mappers.
}

%%%%%%%%%%%%%%%%%%%%%%%%%%%%%%%%%%%%%%%%%%%

% time: Sunday 15:00
% URL: https://pretalx.com/sotm2019/talk/VDUV9A/

\newSmallTimeslot{15:00}
\abstractGHS{%
  Huw Davies%
}{%
  National Trust~-- Managing a Path\linebreak Inventory in OpenStreetMap
}{%
  Towards an Open Paths Standard in OSM for the UK%
}{%
  The talk describes the use of OSM as part of an asset management process using a crowd of National
  Trust staff, volunteers, and the public to maintain a network inventory of an estimated 20\,000\,km
  of paths (both Public Rights of Way and permissive paths). The process pro-actively notifies local
  staff of changes to enable on-ground validation.  The process required the definition, and
  consistent application of, a UK standard for path tagging.%
}

%%%%%%%%%%%%%%%%%%%%%%%%%%%%%%%%%%%%%%%%%%%

% time: Sunday 15:00
% URL: https://pretalx.com/sotm2019/talk/K8N3XY/


\abstractHSO{%
  Jochen Topf%
}{%
  OpenStreetMap Data Processing with PostgreSQL/PostGIS%
}{%
}{%
  The PostgreSQL database with the PostGIS extension is an important instrument in the toolbox of
  anybody working with OSM data. This talks explains the basics of working with the SQL database and
  how it handles geographic data. We will look at getting OSM data in and out of such a database and
  what we can do with the data once it is in there.%
}


\abstractAcademic{%
  Levente Juhász \speakerAffiliation{Florida International University},\linebreak
  Hartwig Hochmair \speakerAffiliation{University of Florida},
  Sen Qiao \speakerAffiliation{Florida International University},\linebreak
  Tessio Novack \speakerAffiliation{GIScience Research Group}
}{%
  Exploring the Effects of Pokémon Go\linebreak Vandalism on OpenStreetMap
}{%
}{%
  This presentation describes the nature and life-cycle of carto-vandalism through a data-driven
  analysis of harmful edits originated from Pokémon Go players. It also assesses how the OSM
  community reacts to vandalism.
}

%%%%%%%%%%%%%%%%%%%%%%%%%%%%%%%%%%%%%%%%%%%

% time: Sunday 15:00
% URL: https://pretalx.com/sotm2019/talk/DKVXZK/


\abstractKHS{%
  Alban Vivert \speakerAffiliation{Nomad Maps/CartONG}%
}{%
  Nomad Maps, an Andean Cartographic Itinerancy by Bike~\noVideo%
}{%
  Nomad Maps Film Documentary
}{%
  Here is the \emph{Nomad Maps documentary}: 5 months, 5200 kilometres across the Andes, 100\,000 metres
  of positive altitude difference, all by bike, to meet the local contributors and projects of the
  OSM mapping community of Colombia, Ecuador, and Peru. An alternative way to see the uses of the
  OpenStreetMap!%
}

%%%%%%%%%%%%%%%%%%%%%%%%%%%%%%%%%%%%%%%%%%%

% time: Sunday 15:30
% URL: https://pretalx.com/sotm2019/talk/KATR7E/

\newSmallTimeslot{15:30}
\abstractHSO{%
  Carolina Ortega Espinosa \speakerAffiliation{Universidad de Antioquia}%
}{%
  Collaborative Cartography of Cycling\linebreak Infrastructure for Route and Thematic Maps in Medellin, Colombia%
}{%
}{%
  The project created from the cooperation between GeoLab (Universidad de Antioquia), and SiCLas
  (group of cyclists), both present in Aburrá Valley (Colombia), proposes a collaborative mapping by
  bicycle users as an urban transport mode. The data generation from existing cycling infra\-structure
  will allow an improvement of the OpenStreetMap database, and an optimisation in route calculation. In addition,
  the incorporation of surveys will allow the generation of thematic maps, such as the association
  of gender with mobility.%
}


\abstractAcademic{%
  A. Yair Grinberger \speakerAffiliation{Heidelberg University},
  Moritz Schott \speakerAffiliation{ibid.},
  Martin Raifer \speakerAffiliation{ibid.},
  Rafael Troilo \speakerAffiliation{ibid.},\linebreak
  Alexander Zipf \speakerAffiliation{ibid.}
}{%
  Analysing the Spatio-Temporal Patterns\linebreak and Impacts of Large-Scale Data\linebreak Production Events in OSM
}{%
}{%
  In this talk, large scale data production events in OpenStreetMap are identified, characterised, and their
  spatio-temporal patterns and impacts are analysed. The results show that remote mapping events
  produce more data today than bulk imports, yet that the former type has a more lasting impact on
  representation, hence pointing towards possible steps for maximising the positive influences of
  events of different types.
}

\newpage

%%%%%%%%%%%%%%%%%%%%%%%%%%%%%%%%%%%%%%%%%%%

% time: Sunday 16:30
% URL: https://pretalx.com/sotm2019/talk/SKVRRL/

\newSmallTimeslot{16:30}
\abstractGHS{%
  Chris Fleming%
}{%
  Notes: Can We Do Better%
}{%
  Experiences and Ideas from the Frontline%
}{%
  An analysis of Notes, based on local experience of managing notes.%
}

%%%%%%%%%%%%%%%%%%%%%%%%%%%%%%%%%%%%%%%%%%%

% time: Sunday 16:30
% URL: https://pretalx.com/sotm2019/talk/XRL7VK/


\abstractHSO{%
  Roland Olbricht%
}{%
  Mapper's Privacy%
}{%
}{%
  OSM's processes are carefully designed to minimise the privacy footprint of the mappers.
  Nonetheless, the principle that any edit shall be attributable means that some data is still
  recorded.
  An overview is given of which data is recorded and which of it becomes available to whom.%
}


\abstractAcademic{%
  Hannah Friedrich \speakerAffiliation{Oregon State University},\linebreak
  Jamon Van Den Hoek \speakerAffiliation{ibid.},
  David Wrathall \speakerAffiliation{ibid.},
  Anna Ballasiotes \speakerAffiliation{ibid.}
}{%
  Development after Displacement:
  Using OSM Data to Measure SDG Indicators at Informal Settlements
}{%
}{%
  There are 250 million refugees and IDPs in informal settlements that are routinely excluded from
  population and settlement datasets as well as Sustainable Development Goals (SDGs) assessments.
  Here, we share results from ongoing research to map and assess SDG indicators at global informal
  settlements using OSM data and satellite imagery. We present a new OSM-driven schema for
  monitoring SDG progress that counters the exclusion of informal settlements from other
  assessments.
}

%%%%%%%%%%%%%%%%%%%%%%%%%%%%%%%%%%%%%%%%%%%

% time: Sunday 16:30
% URL: https://pretalx.com/sotm2019/talk/AV9NWC/

\enlargethispage{2\baselineskip}

\abstractKHS{%
  Simon Poole
}{%
  Updating our Attribution Guidelines~\noVideo%
}{%
}{%
  Most of the existing attribution guidance for OSM derived works dates back to 2012 and was written
  before or around the change of the OSM licence to the ODbL. While there have been relevant
  discussions and rulings by the OSMF LWG (Licence Working Group) since then, there is no easy way
  to find it in one document. The LWG has undertaken to review the existing guidance and update it
  where necessary and is now asking for community input.
}

%%%%%%%%%%%%%%%%%%%%%%%%%%%%%%%%%%%%%%%%%%%

% time: Sunday 16:30
% URL: https://pretalx.com/sotm2019/talk/NZVDN3/


\abstractMathematikonC{%
  Harry Mahardhika Machmud%
}{%
  JOSM Turn Restriction: Improving Data Quality~\noVideo%
}{%
}{%
  Using JOSM Plugins for mapping roads in an advanced way and improving data quality in OpenStreetMap.%
}

%%%%%%%%%%%%%%%%%%%%%%%%%%%%%%%%%%%%%%%%%%%

% time: Sunday 17:00
% URL: https://pretalx.com/sotm2019/talk/DW7WW8/

\newSmallTimeslot{17:00}
\abstractGHS{%
  Martin Lucas-Smith%
}{%
  Is the OSM Data Model Creaking?%
}{%
  OSM Tries to Represent Spaces as Flows (Lines), Resulting in\linebreak Fundamental Compromises. Do We Need
  to Address This?
}{%
  The OSM data model has facilitated rapid growth of community-created geodata which third parties
  can build on. But as more accuracy is needed in routing, cartography, and other uses, is this data
  model good enough? We are trying to represent spaces as flows, which result in fundamental
  compromises and inaccuracies. This talk will discuss real-world cases where this compromise is
  increasingly problematic.%
}

%%%%%%%%%%%%%%%%%%%%%%%%%%%%%%%%%%%%%%%%%%%

% time: Sunday 17:00
% URL: https://pretalx.com/sotm2019/talk/XHGBU7/

\enlargethispage{1\baselineskip}

\abstractHSO{%
  Marc Farra \speakerAffiliation{Development Seed}%
}{%
  Teams for OpenStreetMap%
}{%
}{%
  OSM Teams is a software framework for building team-based applications on top of OpenStreetMap. We
  will present how the software is built, why we think it's a good tool for communities, and how you
  can integrate your application with the framework.%
}


\abstractAcademic{%
  Godwin Yeboah \speakerAffiliation{University of Warwick},
  Rafael Troilo \speakerAffiliation{Heidelberg University},
  Vangelis Pitidis \speakerAffiliation{University of Warwick},
  João Porto de Albuquerque \speakerAffiliation{ibid.}
}{%
  Analysis of OSM Data Quality at\linebreak Different Stages of a Participatory\linebreak Mapping Process~--
  Evidence from\linebreak Informal Urban Settings
}{%
}{%
  This study examines OpenStreetMap data quality at different stages of a participatory mapping
  process developed for understanding inequalities in healthcare access of informal urban residents
  in Africa and Asia. Recent studies have examined quality intrinsically and extrinsically. However,
  in both cases, the data production processes are often not completely transparent to researchers,
  therefore limiting possibilities for systematic data quality analysis of the processes leading to
  OpenStreetMap update.
}

%%%%%%%%%%%%%%%%%%%%%%%%%%%%%%%%%%%%%%%%%%%

% time: Sunday 17:00
% URL: https://pretalx.com/sotm2019/talk/8GY9WF/


\abstractKHS{%
  Joost Schouppe \speakerAffiliation{OSMF}%
}{%
  Local Chapters Congress%
}{%
}{%
  A place where local chapters can meet with each other.%
}

%%%%%%%%%%%%%%%%%%%%%%%%%%%%%%%%%%%%%%%%%%%

% time: Sunday 17:30
% URL: https://pretalx.com/sotm2019/talk/V7NUWP/

\newSmallTimeslot{17:30}
\abstractGHS{%
  Ilya Zverev \speakerAffiliation{Juno Lab}%
}{%
  Broken Promises and Technical\linebreak Difficulties%
}{%
  Oh no, not That API 0.7 Talk Again
}{%
  Our data model is impractical, you know that. Even OGC Simple Features are better. Changesets and
  versions promised easier reverting~-- is it simple yet? We have added a lot of features to API~0.6
  over the past ten years, but should we have? Let's see what went wrong and what we can improve.%
}

%%%%%%%%%%%%%%%%%%%%%%%%%%%%%%%%%%%%%%%%%%%

% time: Sunday 17:30
% URL: https://pretalx.com/sotm2019/talk/KMP9X7/

\newpage

\abstractHSO{%
}{%
  Lightning Talks V%
}{%
}{%
  Come to the welcome desk in the Chemie-Hörsaalgebäude to sign up. See the board there for an up-to-date schedule.
}


\abstractAcademic{%
  Christina Ludwig \speakerAffiliation{Heidelberg University},
  Robert Hecht \speakerAffiliation{Leibniz Institute of Ecological Urban and Regional Development},
  Sven Lautenbach \speakerAffiliation{Heidelberg University},
  Martin Schorcht \speakerAffiliation{Leibniz Institute of Ecological Urban and Regional Development},
  Alexander Zipf \speakerAffiliation{Heidelberg University}
}{%
  Assessing the Completeness of Urban Green Spaces in OpenStreetMap
}{%
}{%
  OpenStreetMap provides a lot of valuable information about urban green spaces, but the numerous
  and conceptually overlapping OSM tags that describe such features lead to spatially heterogenous
  representations in OSM. We developed an exploratory data analysis methodology to identify locally
  relevant OSM tags for mapping green spaces in a specific area and compared the extracted OSM
  features to administrative data to evaluate the level of completeness in regard to urban green
  spaces.
}

%%%%%%%%%%%%%%%%%%%%%%%%%%%%%%%%%%%%%%%%%%%

\newSmallTimeslot{18:00}
\abstractOther{%
}{%
  Poster Session%
}{%
}{%
  More details can be found at page~\pageref{poster-event}.
}{%
  Mathematikon%
}
